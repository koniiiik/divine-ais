\section{Activity, State Machine a Sequence diagramy}

\subsection{Zadanie}
\begin{enumerate} 
\item Vytvorte 2 activity diagrami a  2 state machine diagrami, pričom počet akcií v jednom aktivity diagrame je cca 10 a to isté platí pre state machine diagram, kde počet platí pre stavy.

\item V activity diagramoch chceme, aby ste použili fork, join, merge, decision.

\item State machine - prechody by mali obsahovať spúšťače a ak sa má vykonať nejaká aktivita, tak aj aktivity. V každom stave, ak je to potrebné, uveďte entry a exit aktivity. Využite všetky typy pseudostavov - fork, join, merge, decision -junction nemusíte.
Sequence diagram - popíšte aspoň 2 interakcie sequence diagramom, ktorý obsahuje aspoň 3 účastníkov. Každá posielaná správa musí mať uvedený popis. Použite správny typ správ.
\end{enumerate} 

\clearpage 
\subsection{Activity diagramy}
\subsubsection{Prvy}
\includegraphics[width=\textwidth]{umlet-models/activity_diagram_1.pdf}
\subsubsection{Druhy}
\includegraphics[width=\textwidth]{umlet-models/activity_diagram_2.pdf}

\clearpage 
\subsection{State machine diagramy} 
\subsubsection{Prvy}
\includegraphics[width=\textwidth]{umlet-models/state_machine_diagram1.pdf}
\subsubsection{Druhy}
\includegraphics[width=\textwidth]{umlet-models/state_machine_diagram2.pdf}

\clearpage 
\subsection{Sequence diagramy} 
\subsubsection{Prvy}
\includegraphics[width=\textwidth]{umlet-models/sequence_diagram1.pdf}
\subsubsection{Druhy}
\includegraphics[width=\textwidth]{umlet-models/sequence_diagram2.pdf}

\clearpage
Wow. 


