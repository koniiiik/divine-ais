\section{Use case}

\subsection{Zoznam aktérov}

\begin{description}
    \item[Študent] si vyberá a~zapisuje predmety a~zostavuje rozvrh
    zapisovaním sa na rozvrhové lístky.

    \item[Učiteľ] vyučuje študentov a~prideľuje im hodnotenie.

    \item[Doktorand] je študent vyššieho stupňa, ktorý zároveň vyučuje
    a~hodnotí študentov v~nižších stupňoch štúdia.

    \item[Pracovník študijného oddelenia] zodpovedá za správu študentov
    a~potvrdzuje zápisné listy.

    \item[Tajomník katedry] spravuje rozvrh a~predmety.

    \item[Správca] má neobmedzený prístup do systému.
\end{description}

\subsection{Use case diagram}

\includegraphics[width=\textwidth]{umlet-models/usecase.pdf}

\begin{usecase}{UC-1: Zapíš na predmet}
    \usecasegoal{Zapísanie študenta na predmet.}
    \usecaseprimaryactor{Študent}
    \usecasesecondaryactors{N/A}
    \usecasestartevent{
        \item Študent sa rozhodne zapísať sa na predmet.
    }
    \usecasepreconditions{
        \item Študent ešte nemá daný predmet zapísaný.
    }
    \usecasepostconditions{
        \item Študent má zapísaný predmet v systéme.
    }
    \usecasescenario{
        \item Študent sa v systéme dostane do sekcie Zápis Predmetov
        \item Následne si vyberie predmet a klikne na Potvrdiť.
    }
    \usecaseexcept{Nesplnené prerekvizity}{Študent si dal zapísať predmet,
    ku ktorému nemá splnené všetky prerekvizvity}{
        \item Systém užívateľa upozorní, že nemá splnené prerekvizity
        \item Vráť sa ku kroku 1
    }
    \usecaseexcept{Prekročený limit kreditov}{Študent si dal zapísať
    predmet, avšak súčet kreditov v danom roku prevyšuje maximálny počet
    kreditov za rok.}{
        \item Systém užívateľa upozorní, že počet kreditov by prevýšil maximálny počet kreditov za rok
        \item Vráť sa ku kroku 1
    }
\end{usecase}

\begin{usecase}{UC-2: Pridaj študenta}
    \usecasegoal{Pridanie študenta do systému.}
    \usecaseprimaryactor{Pracovník študijného oddelenia}
    \usecasesecondaryactors{Študent}
    \usecasestartevent{
        \item Pracovník študijného oddelenia sa rozhodne pridať nového študenta do systému.
    }
    \usecasepreconditions{
        \item Pridávaný študent nie je ešte v~systéme zaregistrovaný.
    }
    \usecasepostconditions{
        \item Študent je v~systéme zaregistrovaný s jednoznačným identifikátorom.
    }
    \usecasescenario{
        \item Pracovník študijného oddelenia sa v systéme dostane do sekcie Pridaj študenta
        \item Následne zadá všetky údaje o študentovi, ktoré systém vyžaduje.
        \item Klikne na tlačídlo Potvrdiť.
    }
    \usecaseexcept{Existujúci študent}{Pracovník študijného oddelenia chce
    pridať študenta, ktorý je už v systéme zaregistrovaný.}{
        \item Systém pracovníka upozorní, že tento študent už v systéme zaregistrovaný je
        \item Vráť sa ku kroku 1
    }
    \usecaseexcept{Nesplnené podmienky}{Zadané údaje o študentovi
    vypovedajú o tom, že študent nesplnil podmienky na zápis na školu (do
    systému), ktoré sú zadané v systéme (napr. súčasné štúdium na inej
    škole).}{
        \item Systém pracovníka upozorní, že daný študent nespĺňa podmienky
        \item Vráť sa ku kroku 1
    }
\end{usecase}

\begin{usecase}{UC-3: Zapíš na rozvrhový lístok}
    \usecasegoal{Zapísanie študenta na konkrétny rozvrhový lístok}
    \usecaseprimaryactor{Študent}
    \usecasesecondaryactors{Učiteľ}
    \usecasestartevent{
        \item Študent si vyberie konkrétny rozvrhový lístok, kde bude navštevovať predmet, ktorý je k danému rozvrhovému lístku.
    }
    \usecasepreconditions{
        \item Študent ešte nemá vybraný rozvrhový lístok k danému predmetu.
    }
    \usecasepostconditions{
        \item Študent má vybranú množinu rozvrhových lístkov k danému predmetu.
    }
    \usecasescenario{
        \item Študent sa dostane v systéme do sekcie Vyber rozvrhový lístok
        \item Systém ponúkne študentovi všetky rozvrhové lístky k danému predmetu
        \item Študent si vyberie jeden alebo viac rozvrhových lístkov kliknutím na nich
        \item Ak má vybrané všetky, klikne na tlačidlo Potvrdiť.
    }
    \usecaseexcept{Maximálny počet študentov}{Študent si vybral rozvrhový
    lístok, kde je už maximálny počet študentov.}{
        \item Systém študenta upozorní, že pri tomto rozvrhovom lístku je už zapísaný maximálny počet 	študentov.
        \item Vráť sa ku kroku 1
    }
    \usecaseexcept{Kolízia s iným rozvrhovým lístkom}{Študent si vybral
    rozvrhový lístok, ktorého čas sa kryje s nejakým iným rozvrhovým
    lístkom iného predmetu, ktorý má zapísaný.}{
        \item Systém študenta upozorní, aby si vybral iný rozvrhový lístok
        \item Vráť sa ku kroku 1
    }
\end{usecase}

\begin{usecase}{UC-4: Ohodnoť študenta}
    \usecasegoal{Zadať do systému hodnotenie študenta.}
    \usecaseprimaryactor{Učiteľ}
    \usecasesecondaryactors{Študent}
    \usecasestartevent{
        \item Učiteľ chce zadať do systému hodnotenie študenta k predmetu, ktorý vyučuje.
    }
    \usecasepreconditions{
        \item Učiteľ ešte nezadal hodnotenie študenta do systému.
    }
    \usecasepostconditions{
        \item V systéme sa nachádza hodnotenie študenta k predmetu.
    }
    \usecasescenario{
        \item Učiteľ sa dostane v systéme do sekcie Zadaj hodnotenie
        \item Systém ponúkne učiteľovi zoznam predmetov, ktoré vyučuje.
        \item Učiteľ si vyberie predmet, následne mu systém ponúkne zoznam študentov, ktorý majú predmet zapísaný.
        \item Učiteľ si vyberie študenta.
        \item Učiteľ zadá do systému hodnotenie A-Fx a klikne na tlačidlo potvrdiť.
    }
    \usecaseexcept{Hodnotenie už existuje.}{Učiteľ chce zadať hodnotenie študenta, ktorý v systéme už hodnotenie má.}{
        \item Systém učiteľa upozorní, že už bol z tohto predmetu hodnotený.
        \item Vráť sa ku kroku 4
    }
\end{usecase}

\begin{usecase}{UC-5: Nastav nový semester}
    \usecasegoal{Pridať do systému údaje o novom semestri.}
    \usecaseprimaryactor{Správca}
    \usecasesecondaryactors{N/A}
    \usecasestartevent{
        \item Administrátor chce pridať do systému nový semester.
    }
    \usecasepreconditions{
        \item Semester ešte nie je v systéme.
    }
    \usecasepostconditions{
        \item V systéme sa nachádza záznam o semestri s príslušnými údajmi.
    }
    \usecasescenario{
        \item Administrátor sa v systéme dostane do sekcie Zadaj hodnotenie
        \item Následne zadá všetky údaje o semestri (začiatok semestra, termíny vyučovania a skúškového obdobia atď.)
        \item Klikne na tlačidlo Potvrdiť.
    }
    \usecaseexcept{Semester už existuje.}{Administrátor chcel pridať do systému semester, ktorý sa v ňom už nachádza.}{
        \item Systém administrátora upozorní, že takýto semester sa už nachádza v systéme.
        \item Vráť sa ku kroku 2
    }
    \usecaseexcept{Neplatné údaje o semestri}{Administrátor zadal neplatné údaje o semestri (zlý rok, prekrývajúc sa termíny jednotlivých období semestra atď.)}{
        \item Systém administrátora upozorní na konkrétne konflikty.
        \item Vráť sa ku kroku 2
    }
\end{usecase}

\begin{usecase}{UC-6: Vytvoriť rozvrh}
    \usecasegoal{vytvoriť rozvrh hodín a zadať ho do systému.}
    \usecaseprimaryactor{Tajomník katedry}
    \usecasesecondaryactors{N/A}
    \usecasestartevent{
        \item Tajomník katedry chce pridať do systému rozvrh hodín pre študijnú skupinu.
    }
    \usecasepreconditions{
        \item Cieľová študijná skupina ešte nemá pridelený rozvrh.
    }
    \usecasepostconditions{
        \item Cieľová študijná skupina má pridelený rozvrh.
    }
    \usecasescenario{
        \item Tajomník katedry sa v systéme dostane do sekcie Pridaj rozvrh
        \item Následne vytvorí rozvrh hodín v príslušnom užívateľskom prostredí.
        \item Klikne na tlačidlo Potvrdiť.
    }
    \usecaseexcept{Rozvrh pre danú skupinu študentov už existuje.}{Tajomník katedry chcel pridať do systému rozvrh hodín študijnej skupiny, ktorá už rozvrh hodín pridelený má.}{
        \item Systém tajomníka katedry upozorní, že daná študijná skupina 
        \item Vráť sa ku kroku 2
    }
\end{usecase}

\begin{usecase}{UC-7: Potvrď zápis predmetov}
    \usecasegoal{Pridať do systému údaje o novom semestri.}
    \usecaseprimaryactor{Pracovník študijného oddelenia}
    \usecasesecondaryactors{Študent}
    \usecasestartevent{
        \item Študent si chce dať potvrdiť zápis predmetov na študijnom oddelení.
    }
    \usecasepreconditions{
        \item Študent má zapísané predmety v systéme, ale nemá ich doposiaľ potvrdené.
    }
    \usecasepostconditions{
        \item Študent má zapísané predmety aj potvrdené.
    }
    \usecasescenario{
        \item Pracovník študijného oddelenia sa v systéme dostane do sekcie Potvrď zápis predmetov
        \item Systém vykoná kontrolu zapísaných predmetov a overí, či platia všetky podmienky pre zápis predmetov.
        \item Klikne na tlačidlo Potvrdiť.
    }
    \usecaseexcept{Študent už má potvrdený zápis.}{Študent si dal potvrdiť zápis predmetov, avšak zápis bol už vykonaný predtým.}{
        \item Systém pracovníka študijného oddelenia upozorní, že už má zápis potvrdený.
        \item Ukonči akciu.
    }
    \usecaseexcept{Nesplnené podmienky pre potvrdenie.}{Študent si nemôže dať potvrdiť zápis, pretože nie sú splnené všetky podmienky (nekonzistentnosť s indexom, málo kreditov atď.)}{
        \item Pracovník študijného oddelenia odporučí študentovi čo má ešte opraviť v zápise, aby si mohol dať zápis potvrdiť.
        \item Ukonči akciu.
    }
\end{usecase}

\begin{usecase}{UC­8: Prezri hodnotenie}
    \usecasegoal{Študentovi sa zobrazia všetky existujúce hodnotenia k jeho predmetom.}
    \usecaseprimaryactor{Študent}
    \usecasesecondaryactors{N/A}
    \usecasestartevent{
        \item Študent si chce prezrieť svoje doterajšie hodnotenia.
    }
    \usecasepreconditions{
        \item Študent má zapísané predmety.
        \item Vyučujúci zadal hodnotenie študentovi aspoň z jedného predmetu, ktorý má študent zapísaný.  
    }
    \usecasepostconditions{
        \item N/A 
    }
    \usecasescenario{
        \item Študent sa dostane do sekcie Evidencie štúdia - Hodnotenia a priemery. 
        \item Systém načíta jeho všetky zapísané predmety. Zobrazí k nim príslušné hodnotenie ak existuje a ak nie, tak vypíše N/A. 
    }
    \usecaseexcept{Študent nemá zapísaný predmet.}{Študent si chcel zobraziť hodnotenia, avšak ešte nemá zapísané predmety.}{
        \item Systém študenta upozorní, že si najprv musí zapísať predmety. 
        \item Ukonči akciu.
    }
    \usecaseexcept{Neexistuje hodnotenie študenta.}{Študent má zapísaný aspoň jeden predmet, ale vyučujúci mu ešte nezadal hodnotenie. }{
        \item Systém študenta upozorní, že dosiaľ nemá žiadne hodnotenia. 
        \item Zobraz všetky zapísané predmety. 
        \item Ukonči akciu. 
    }
\end{usecase}
 
\begin{usecase}{UC­9: Prezri rozvrh}
    \usecasegoal{Študentovi sa zobrazí jeho rozvrh.}
    \usecaseprimaryactor{Študent}
    \usecasesecondaryactors{N/A}
    \usecasestartevent{
        \item Študent si chce prezrieť svoj aktuálny rozvrh.
    }
    \usecasepreconditions{
        \item Študent má zapísané predmety.
        \item Je vytvorený rozvrh na aktuálny semester.
        \item Študent má zapísaný aspoň jeden rozvrhový lístok.  
    }
    \usecasepostconditions{
        \item N/A
    }
    \usecasescenario{
        \item Študent sa dostane do sekcie Evidencie štúdia - Rozvrh. 
        \item Systém načíta všetky rozvrhové lístky, ktoré má študent zapísané v aktuálnom semestri a zobrazí ich ako kalendár. 
    }
    \usecaseexcept{Neexistuje rozvrh.}{Tajomník nevytvoril rozvrh k aktuálnemu semestru.}{
        \item Systém študenta upozorní, že zatiaľ nie je možné zobraziť rozvrhové lístky, lebo ešte nebol vytvorený rozvrh pre aktuálny semester. 
        \item Systém zobrazí prázdny kalendár. 
        \item Ukonči akciu. 
    }
    \usecaseexcept{Neexistuje zapísaný rozvrhový lístok.}{Študent ešte nemá zapísaný rozvrhový lístok.}{
        \item Systém študenta upozorní, že si najprv musí zapísať rozvrhový lístok. 
        \item Systém zobrazí prázdny kalendár. 
        \item Ukonči akciu. 
    }
\end{usecase}
	
\begin{usecase}{UC­10: Pridaj nový predmet}
    \usecasegoal{Pridať do systému nový predmet.}
    \usecaseprimaryactor{Tajomník}
    \usecasesecondaryactors{Učiteľ}
    \usecasestartevent{
        \item Tajomník chce pridať nový predmet do ponuky predmetov.
    }
    \usecasepreconditions{
        \item Učiteľ súhlasil s výukou predmetu.
    }
    \usecasepostconditions{
        \item Predmet je evidovaný v systéme, má jednoznačný identifikátor a je viditeľný v ponuke predmetov.
    }
    \usecasescenario{
        \item Tajomník sa v systéme dostane do sekcie pridaj predmet a vyplní základné údaje ako názov, kód, učiteľ predmetu, atď. .
        \item Systém upozorní učiteľa (ktorého zvolil ako vyučujúceho tajomník v kroku 1), aby súhlasil s výukou predmety a doplnil ďalšie údaje ako sylabus, typ výuky, atď. .  
        \item Učiteľ potvrdí údaje a súhlasí s pridaním predmetu.
        \item Tajomník potvrdí údaje a súhlasí s pridaním predmetu. 
    }
    \usecaseexcept{Učiteľ nesúhlasí.}{Učiteľ nesúhlasí s výukou pridávaného predmetu.}{
        \item Systém upozorní tajomníka, že učiteľ nesúhlasil s výukou predmetu.
        \item Učiteľ sa s tajomníkom dohodnú, či bude predmet vytvorený. 
        \item Ak áno, pokračuje sa od kroku 2. Ak nie, tak sa všetky doposiaľ vyplnené údaje zrušia a predmet sa do systému nepridá. 
        \item Ukonči akciu. 
    }
    \usecaseexcept{Tajomník nesúhlasí.}{Tajomník klikol na tlačidlo nesúhlasím v kroku 4.}{
        \item Tajomník zadá dôvody, pre ktoré odmietol predmet pridať. 
        \item Učiteľ buď upraví údaje a pokračuje sa krokom 3, alebo sa po dohode s tajomníkom pridávanie predmetu zruší, ako v prípade Výnimka 1 - Krok 3. 
    }
\end{usecase}

\begin{usecase}{UC­11: Pridaj externého študenta}
    \usecasegoal{Pridanie externého študenta do systému.}
    \usecaseprimaryactor{Pracovník študijného oddelenia}
    \usecasesecondaryactors{Externý študent}
    \usecasestartevent{
        \item Pracovník študijného oddelenia sa rozhodne pridať nového externého študenta do systému.
    }
    \usecasepreconditions{
        \item Pridávaný externý študent nie je ešte v systéme zaregistrovaný.
    }
    \usecasepostconditions{
        \item Externý študent je v systéme zaregistrovaný s jednoznačným identifikátorom.
    }
    \usecasescenario{
        \item Pracovník študijného oddelenia sa v systéme dostane do sekcie Pridaj externého študenta. 
        \item Následne zadá všetky údaje o externom študentovi, ktoré systém vyžaduje.
        \item Klikne na tlačidlo Potvrdiť.
    }
    \usecaseexcept{Existujúci externý študent.}{Pracovník študijného oddelenia chce pridať externého študenta, ktorý je už v systéme zaregistrovaný.}{
        \item Systém pracovníka upozorní, že tento externý študent už v systéme zaregistrovaný je. 
        \item Vráť sa ku kroku 1. 
    }
    \usecaseexcept{Nesplnené podmienky.}{Niektoré údaje zadané pracovníkom študíjneho oddelenia nespĺňajú podmienky. }{
        \item Systém pracovníka upozorní, že daný externý študent nespĺňa podmienky.
        \item Vráť sa ku kroku 1. 
    }
\end{usecase}
 
\begin{usecase}{UC­12: Nastav harmonogram semestra}
    \usecasegoal{Pridať do systému dôležité termíny semestra.}
    \usecaseprimaryactor{Administrátor}
    \usecasesecondaryactors{N/A}
    \usecasestartevent{
        \item Administrátor chce zadať dôležité termíny semestra, napríklad pri vytváraní nového semestra.
    }
    \usecasepreconditions{
        \item Daný semester ešte nemá nastavené termíny.
    }
    \usecasepostconditions{
        \item Termíny sú zadané v systéme a ďalej sa využívajú, ako napríklad obdobie zápisu.
    }
    \usecasescenario{
        \item Administrátor sa v systéme dostane do sekcie Semester - Zadaj harmonogram semestra. 
        \item Následne administrátor zadá všetky dôležité termíny semestra, ako napríklad skúškové obdobie, obdobie zápisu, prázdniny, atď. .
    }
    \usecaseexcept{Termíny už existujú.}{Administrátor chce nastaviť termíny semestru, ktorý už má zadaný harmonogram.}{
        \item Systém administrátora upozorní, že tento semester už má zadané termíny a opýta sa ho, či chce naozaj pokračovať, keďže zmena ovplyvní celý systém. 
        \item Ak administrátor nechce ďalej pokračovať, tak systém ukončí akciu. Ak administrátor chce ďalej pokračovať, tak sa vykoná krok 2. 
        \item Systém upozorní všetkých užívateľov systému, ktorých sa zmena týka. 
        \item Ukonči akciu.
    }
\end{usecase}

