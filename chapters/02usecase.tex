\section{Use case}

\subsection{Zoznam aktérov}

\begin{description}
    \item[Študent] si vyberá a~zapisuje predmety a~zostavuje rozvrh
    zapisovaním sa na rozvrhové lístky.

    \item[Učiteľ] vyučuje študentov a~prideľuje im hodnotenie.

    \item[Doktorand] je študent vyššieho stupňa, ktorý zároveň vyučuje
    a~hodnotí študentov v~nižších stupňoch štúdia.

    \item[Pracovník študijného oddelenia] zodpovedá za správu študentov
    a~potvrdzuje zápisné listy.

    \item[Tajomník katedry] spravuje rozvrh a~predmety.

    \item[Správca] má neobmedzený prístup do systému.
\end{description}

\subsection{Use case diagram}

\includegraphics[width=\textwidth]{umlet-models/usecase.pdf}

\begin{usecase}{UC-1: Zapíš na predmet}
    \usecasegoal{Zapísanie študenta na predmet.}
    \usecaseprimaryactor{Študent}
    \usecasesecondaryactors{N/A}
    \usecasestartevent{
        \item Študent sa rozhodne zapísať sa na predmet.
    }
    \usecasepreconditions{
        \item Študent ešte nemá daný predmet zapísaný.
    }
    \usecasepostconditions{
        \item Študent má zapísaný predmet v systéme.
    }
    \usecasescenario{
        \item Študent sa v systéme dostane do sekcie Zápis Predmetov
        \item Následne si vyberie predmet a klikne na Potvrdiť.
    }
    \usecaseexcept{Nesplnené prerekvizity}{Študent si dal zapísať predmet,
    ku ktorému nemá splnené všetky prerekvizvity}{
        \item Systém užívateľa upozorní, že nemá splnené prerekvizity
        \item Vráť sa ku kroku 1
    }
    \usecaseexcept{Prekročený limit kreditov}{Študent si dal zapísať
    predmet, avšak súčet kreditov v danom roku prevyšuje maximálny počet
    kreditov za rok.}{
        \item Systém užívateľa upozorní, že počet kreditov by prevýšil maximálny počet kreditov za rok
        \item Vráť sa ku kroku 1
    }
\end{usecase}

\begin{usecase}{UC-2: Pridaj študenta}
    \usecasegoal{Pridanie študenta do systému.}
    \usecaseprimaryactor{Pracovník študijného oddelenia}
    \usecasesecondaryactors{Študent}
    \usecasestartevent{
        \item Pracovník študijného oddelenia sa rozhodne pridať nového študenta do systému.
    }
    \usecasepreconditions{
        \item Pridávaný študent nie je ešte v~systéme zaregistrovaný.
    }
    \usecasepostconditions{
        \item Študent je v~systéme zaregistrovaný s jednoznačným identifikátorom.
    }
    \usecasescenario{
        \item Pracovník študijného oddelenia sa v systéme dostane do sekcie Pridaj študenta
        \item Následne zadá všetky údaje o študentovi, ktoré systém vyžaduje.
        \item Klikne na tlačídlo Potvrdiť.
    }
    \usecaseexcept{Existujúci študent}{Pracovník študijného oddelenia chce
    pridať študenta, ktorý je už v systéme zaregistrovaný.}{
        \item Systém pracovníka upozorní, že tento študent už v systéme zaregistrovaný je
        \item Vráť sa ku kroku 1
    }
    \usecaseexcept{Nesplnené podmienky}{Zadané údaje o študentovi
    vypovedajú o tom, že študent nesplnil podmienky na zápis na školu (do
    systému), ktoré sú zadané v systéme (napr. súčasné štúdium na inej
    škole).}{
        \item Systém pracovníka upozorní, že daný študent nespĺňa podmienky
        \item Vráť sa ku kroku 1
    }
\end{usecase}

\begin{usecase}{UC-3: Zapíš na rozvrhový lístok}
    \usecasegoal{Zapísanie študenta na konkrétny rozvrhový lístok}
    \usecaseprimaryactor{Študent}
    \usecasesecondaryactors{Učiteľ}
    \usecasestartevent{
        \item Študent si vyberie konkrétny rozvrhový lístok, kde bude navštevovať predmet, ktorý je k danému rozvrhovému lístku.
    }
    \usecasepreconditions{
        \item Študent ešte nemá vybraný rozvrhový lístok k danému predmetu.
    }
    \usecasepostconditions{
        \item Študent má vybranú množinu rozvrhových lístkov k danému predmetu.
    }
    \usecasescenario{
        \item Študent sa dostane v systéme do sekcie Vyber rozvrhový lístok
        \item Systém ponúkne študentovi všetky rozvrhové lístky k danému predmetu
        \item Študent si vyberie jeden alebo viac rozvrhových lístkov kliknutím na nich
        \item Ak má vybrané všetky, klikne na tlačidlo Potvrdiť.
    }
    \usecaseexcept{Maximálny počet študentov}{Študent si vybral rozvrhový
    lístok, kde je už maximálny počet študentov.}{
        \item Systém študenta upozorní, že pri tomto rozvrhovom lístku je už zapísaný maximálny počet 	študentov.
        \item Vráť sa ku kroku 1
    }
    \usecaseexcept{Kolízia s iným rozvrhovým lístkom}{Študent si vybral
    rozvrhový lístok, ktorého čas sa kryje s nejakým iným rozvrhovým
    lístkom iného predmetu, ktorý má zapísaný.}{
        \item Systém študenta upozorní, aby si vybral iný rozvrhový lístok
        \item Vráť sa ku kroku 1
    }
\end{usecase}

\begin{usecase}{UC-4: Ohodnoť študenta}
    \usecasegoal{Zadať do systému hodnotenie študenta.}
    \usecaseprimaryactor{Učiteľ}
    \usecasesecondaryactors{Študent}
    \usecasestartevent{
        \item Učiteľ chce zadať do systému hodnotenie študenta k predmetu, ktorý vyučuje.
    }
    \usecasepreconditions{
        \item Učiteľ ešte nezadal hodnotenie študenta do systému.
    }
    \usecasepostconditions{
        \item V systéme sa nachádza hodnotenie študenta k predmetu.
    }
    \usecasescenario{
        \item Učiteľ sa dostane v systéme do sekcie Zadaj hodnotenie
        \item Systém ponúkne učiteľovi zoznam predmetov, ktoré vyučuje.
        \item Učiteľ si vyberie predmet, následne mu systém ponúkne zoznam študentov, ktorý majú predmet zapísaný.
        \item Učiteľ si vyberie študenta.
        \item Učiteľ zadá do systému hodnotenie A-Fx a klikne na tlačidlo potvrdiť.
    }
    \usecaseexcept{Hodnotenie už existuje.}{Učiteľ chce zadať hodnotenie študenta, ktorý v systéme už hodnotenie má.}{
        \item Systém učiteľa upozorní, že už bol z tohto predmetu hodnotený.
        \item Vráť sa ku kroku 4
    }
\end{usecase}

\begin{usecase}{UC-5: Nastav nový semester}
    \usecasegoal{Pridať do systému údaje o novom semestri.}
    \usecaseprimaryactor{Správca}
    \usecasesecondaryactors{N/A}
    \usecasestartevent{
        \item Administrátor chce pridať do systému nový semester.
    }
    \usecasepreconditions{
        \item Semester ešte nie je v systéme.
    }
    \usecasepostconditions{
        \item V systéme sa nachádza záznam o semestri s príslušnými údajmi.
    }
    \usecasescenario{
        \item Administrátor sa v systéme dostane do sekcie Zadaj hodnotenie
        \item Následne zadá všetky údaje o semestri (začiatok semestra, termíny vyučovania a skúškového obdobia atď.)
        \item Klikne na tlačidlo Potvrdiť.
    }
    \usecaseexcept{Semester už existuje.}{Administrátor chcel pridať do systému semester, ktorý sa v ňom už nachádza.}{
        \item Systém administrátora upozorní, že takýto semester sa už nachádza v systéme.
        \item Vráť sa ku kroku 2
    }
    \usecaseexcept{Neplatné údaje o semestri}{Administrátor zadal neplatné údaje o semestri (zlý rok, prekrývajúc sa termíny jednotlivých období semestra atď.)}{
        \item Systém administrátora upozorní na konkrétne konflikty.
        \item Vráť sa ku kroku 2
    }
\end{usecase}
